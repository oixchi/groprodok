\chapter{Aufgabenstellung}
\section{Aufgabenanalyse}
\subsection{Restriktionen und Annahmen}
Aus der Aufgabenstellung werden mehrere Bedingungen ersichtlich, welche von der Software beachtet werden müssen.
\subsection{Formate}
\subsubsection{Die Eingabedatei}
Die Eingabedatei setzt sich aus mehreren Kommentaren und Informationen zusammen. Ein Kommentar wird hier durch ein \# eingeleitet. Die Kommentare sind für die Simulation des Programms nicht von Belang, sie finden nur teilweise in der Ausgabe Verwendung als Information. Die übrigen Zeilen beinhalten die relevanten Eingabeinformationen für die Weiterverarbeitung. Der Aufbau der Datei gestaltet sich wie folgt:
\begin{itemize}
	\item 
	\item 
\end{itemize}
\subsubsection{Die Ausgabedatei}
\subsection{Mögliche Probleme}
Mögliche Ursachen, die die Verarbeitung der Datei verhindern, oder zu einem falschen Ergebnis führen:
\begin{itemize}
	\item 
\end{itemize}

\chapter{Verbale Verfahrensbeschreibung}
\section{Klasse Main}
Die Klasse Main kümmert sich um den Ablauf des Programms.
\section{Klasse Einlesen}
\section{Klasse Ausgabe}
Die Klasse ermöglicht das Erstellen des Ausgabedatei nach Vorgabe der Prüfung.
\section{Ablauf}

\chapter{Problemfälle}
\section{Eingabefehler}
Eingabefehler können in diversen Variationen auftreten. Sobald die Vorgaben für den Aufbau des Eingabedokumentes nicht eingehalten werden, wird das Programm mit einem Fehler abgebrochen.
\section{Laufzeit}

\chapter{Abweichungen zum Konzept}
Im Rahmen der ersten Bearbeitung der Aufgabenstellung musste ein Konzept erstellt werden, welches die Programmstruktur und einen möglichen Ablauf beinhalten sollte. Bei der näheren Bearbeitung der Aufgabe ist jedoch aufgefallen, dass die Konzeption teilweise nicht vollständig war und einige Elemente nicht benötigt wurden. Aus dem Vergleich von Konzeption und Endprodukt ergeben sich folgende Änderungen:
\begin{itemize}
	\item 
	\item 
\end{itemize}

\chapter{Programmkonzeption}
\section{Klassendiagramm}
\section{Sequenzdiagramm}

\chapter{Tests}
\section{IHK Beispiele}
\section{Eigenständige Tests}
\section{Fehlerfälle}

\chapter{Benutzeranleitung}
\section{Systemvoraussetzungen}
\section{Benutzeranleitung}


\chapter{Zusammenfassung und Ausblick}